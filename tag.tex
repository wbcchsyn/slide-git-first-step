% This is part of the ``Git First Step''.
% Copyright 2014, 2020 Yoshida Shin.
% See the file slide.tex for copying conditions.

\begin{frame}[t]{tag}{}

  commit に移動しない名前をつける
  \vspace{4ex}

  \onslide*<1,5>{
    現在の tag 一覧を表示
    \vspace{2ex}

    \$ git tag
    \vspace{2ex}

    tag は無いので何も表示されない
  }

  \onslide*<2>{
    tag を作成
    \vspace{2ex}

    \$ git tag v0.0.1
    \vspace{2ex}

    HEAD の向いている commit に対して v0.0.1 という tag をつける
  }

  \onslide*<3>{
    \$ git tag
    \vspace{2ex}

    tag 一覧を表示
    \vspace{2ex}

    \code{v0.0.1}
  }

  \onslide*<4>{
    \$ git tag -d v0.0.1
    \vspace{2ex}

    tag v0.0.1 を削除
  }
\end{frame}


\begin{frame}[t]{tag}{コマンド補足}

  \begin{itemize}
  \item git tag \textit{tag-name} [\textit{commit}]

    で commit を指す tag を作成

    commit は branch, HEAD, ハッシュ値等で指定

    commit のデフォルトは HEAD
    \vspace{2ex}

  \item tag を remote repository へ転送するには

    git push \textit{URI} tag \textit{tag-name}
    \vspace{2ex}

  \item remote repository の tag を削除するには

    git push \textit{URI} :refs/tags/\textit{tag-name}
    \vspace{2ex}

  \item remote repository の tag も git fetch {\dhyphen}all -p で取得可能
    \vspace{2ex}

  \item merge や checkout の commit 指定に tag も使用可能

    ただし、その場合 HEAD が branch から外れるので注意
  \end{itemize}

\end{frame}
