% This is part of the ``Git First Step''.
% Copyright 2014, 2020 Yoshida Shin.
% See the file slide.tex for copying conditions.

\begin{frame}[t]{衝突 (一旦停止)}{更新が競合したら}

  merge を一旦停止して、解決してから 再 merge
  \vspace{4ex}

  \onslide*<1>{
    準備
    \vspace{2ex}

    \$ cd \~{}/local\_rep

    \$ git branch conflict1

    \$ git branch conflict2
  }

  \onslide*<2>{
    conflict1 で更新
    \vspace{2ex}

    \$ git checkout conflict1

    \$ echo ``conflict1'' {\textgreater}{\textgreater} f

    \$ git add f

    \$ git commit -m 'Update 1.'
  }

  \onslide*<3>{
    conflict2 で更新
    \vspace{2ex}

    \$ git checkout conflict2

    \$ echo ``conflict2'' {\textgreater}{\textgreater} f

    \$ git add f

    \$ git commit -m 'Update 2.'
  }

  \onslide*<4>{
    conflict2 で conflict1 と merge (衝突)
    \vspace{2ex}

    \$ git merge conflict1
  }

  \onslide*<5>{
    merge を一旦停止
    \vspace{2ex}

    \$ git merge {\dhyphen}abort
  }

  \onslide*<6>{
    衝突しない様に編集、commit してから再度 merge
    \vspace{2ex}

    \$ vi f

    \$ git add f

    \$ git commit -m 'Escape conflict.'

    \$ git merge conflict1
  }
\end{frame}


\begin{frame}[t]{中で起きている事}{更新が競合したら}

  \begin{columns}

    \begin{narrowcolumn}

      \begin{block}{workspace}
        c d e
        \alert<8-9>{f}
        g h i
      \end{block}

      \begin{block}{cached (index)}
        c d e
        \alert<8-9>{f}
        g h i
      \end{block}

    \end{narrowcolumn}

    \begin{widecolumn}

      \begin{repository}{repository}
        \onslide<14->{
          \node (c13) [commit, rectangle split, rectangle split parts=2] at (3em, 14ex){
            \commitmessage{Merge bra...}
            \nodepart{second}{conflict2}
          };
        }

        \onslide*<14->{
          \node (c12) [commit, rectangle split, rectangle split parts=1] at (3em, 8ex){
            \commitmessage{Escape conflict.}
          };
        }

        \onslide*<12-13>{
          \node (c12) [commit, rectangle split, rectangle split parts=2] at (3em, 10ex){
            \commitmessage{Escape conflict.}
            \nodepart{second}{conflict2}
          };
        }

        \onslide*<6-11>{
          \node (c11) [commit, rectangle split, rectangle split parts=2] at (3em, 6ex){
            \commitmessage{Update 2.}
            \nodepart{second}{conflict2}
          };
        }

        \onslide*<12->{
          \node (c11) [commit, rectangle split, rectangle split parts=1] at (3em, 4ex){
            \commitmessage{Update 2.}
          };
        }

        \onslide*<4->{
          \node (c10) [commit, rectangle split, rectangle split parts=2] at (-3em, 6ex){
            \commitmessage{Update 1.}
            \nodepart{second}{conflict1}
          };
        }

        \node (c9) [commit, rectangle split, rectangle split parts=1] at (0em, 0ex){
          \commitmessage{Merge remote-tracking branch ...}
        };

        \onslide*<14->{\draw (c12) -- (c13);}
        \onslide*<14->{\draw (c10) -- (c13);}
        \onslide*<12->{\draw (c11) -- (c12);}
        \onslide*<6->{\draw (c9) -- (c11);}
        \onslide*<4->{\draw (c9) -- (c10);}
      \end{repository}

    \end{widecolumn}

  \end{columns}
  \vspace{2ex}

  \onslide*<2>{
    \$ cd \~{}/local\_rep

    \$ git branch conflict1

    \$ git branch conflict2
  }

  \onslide*<3-4>{
    \$ git checkout conflict1

    \$ echo ``conflict1'' {\textgreater}{\textgreater} f

    \$ git add f

    \$ git commit -m 'Update1'
  }

  \onslide*<5-6>{
    \$ git checkout conflict2

    \$ echo ``conflict2'' {\textgreater}{\textgreater} f

    \$ git add f

    \$ git commit -m 'Update2'
  }

  \onslide*<7-8>{
    \$ git merge conflict1

    workspace に無理矢理両方の更新を適用した状態
  }

  \onslide*<9-10>{
    \$ git merge {\dhyphen}abort

    前回の git merge を行う前の状態に戻す
  }

  \onslide*<11-12>{
    \$ vi f

    \$ git add f

    \$ git commit -m 'Escape conflict.'

    conflict を起こさないように f を修正
  }

  \onslide*<13-14>{
    \$ git merge conflict1

    merge 成功
  }
\end{frame}
