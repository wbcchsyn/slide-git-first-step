% This is part of the ``Git First Step''.
% Copyright 2014, 2020 Yoshida Shin.
% See the file slide.tex for copying conditions.

\begin{frame}<1-4>[t]{branch (一覧、作成)}{branch を使用する}

  branch の一覧表示、作成
  \vspace{4ex}

  \onslide*<1>{
    現在の branch を確認
    \vspace{2ex}

    \$ git branch

    \code{* master}
  }

  \onslide*<2>{
    新 branch の作成
    \vspace{2ex}

    \$ git branch develop
  }

  \onslide*<3>{
    現在の branch を確認
    \vspace{2ex}

    \$ git branch

    \code{
      * master

      \onslide<99>{* }develop
    }
  }


  \onslide*<4>{
    repository の更新を行ってみる
    \vspace{2ex}

    \$ git add c

    \$ git commit -m 'Add c.'
  }

\end{frame}


\begin{frame}[t]{中で起きている事}{branch を使用する}

  \begin{columns}

    \begin{narrowcolumn}

      \begin{block}{workspace}
        c d
      \end{block}

      \begin{block}{cached (index)}
        \onslide<8->{c} d
      \end{block}

    \end{narrowcolumn}

    \begin{halfcolumn}

      \begin{repository}{repository}

        \onslide<12->{
          \node (c3) [commit, rectangle split, rectangle split parts=2] at (0em, 14ex) {
            \commitmessage{Add c.}
            \nodepart{second}{\onslide<12->{master{\HEAD}}}
          };
        }

        \onslide*<10-11>{
          \node (c3) [commit, rectangle split, rectangle split parts=1] at (0em, 14ex) {
            \commitmessage{Add c.}
          };
        }

        \node (c2) [commit, rectangle split, rectangle split parts=2] at (0em, 6ex) {
          \commitmessage{Rename a to d and delete b.}
          \nodepart{second}{\onslide*<-11>{master{\HEAD}} \onslide*<5->{develop}}
        };

        \node (c1) [commit] at (0em, 0ex) {
          \commitmessage{Add a and b.}
        };

        \onslide<10->{\draw (c2) -- (c3);}
        \draw (c1) -- (c2);
      \end{repository}

    \end{halfcolumn}

  \end{columns}
  \vspace{2ex}

  \onslide*<2-3>{
    \$ git branch
    \vspace{2ex}

    \onslide*<2>{現在の branch 一覧を表示}
    \onslide*<3>{
      \code{* master}

      * は HEAD が向いている branch の印
    }
  }

  \onslide*<4-5>{
    \$ git branch develop
    \vspace{2ex}

    HEAD と同じ commit を指す branch を作成
  }

  \onslide*<6>{
    \$ git branch
    \vspace{2ex}

    \code{
      * master

      \onslide<99>{* }develop
    }
  }

  \onslide*<7-8>{
    \$ git add c
    \vspace{2ex}

    c を cached へコピー
  }

  \onslide*<9-13>{
    \$ git commit -m 'Add c.'
    \vspace{2ex}

    \onslide*<9-10>{cached を取り込み、message を添えて、現在の HEAD を親とする commit を作成}
    \onslide*<11-12>{
      HEAD が branch を指している時は

      その branch の向き先を新しい commit へ変更
    }
    \onslide*<13>{HEAD が指しているのは master だけなので、develop は放置}
  }

\end{frame}


\begin{frame}[t]{git branch コマンド補足 (一覧、作成)}{branch を使用する}

  \begin{itemize}
  \item git branch [-a]

    で現在の branch 一覧を表示

    -a オプションをつけると、remote branch\footnote{後述} も表示する
    \vspace{2ex}

  \item git branch \textit{branch-name} [\textit{start-point}]

    で start-point をさす branch を作成

    start-point は branch, HEAD, ハッシュ値\footnote{後述} 等で指定

    start-point のデフォルト値は HEAD
  \end{itemize}

\end{frame}
