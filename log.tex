% Copyright 2014 Yoshida Shin
% 
% This is part of "Git First Step."
% 
% This program is free software: you can redistribute it and/or modify
%     it under the terms of the GNU General Public License as published by
%     the Free Software Foundation, either version 3 of the License, or
%     (at your option) any later version.
% 
%     This program is distributed in the hope that it will be useful,
%     but WITHOUT ANY WARRANTY; without even the implied warranty of
%     MERCHANTABILITY or FITNESS FOR A PARTICULAR PURPOSE.  See the
%     GNU General Public License for more details.
% 
%     You should have received a copy of the GNU General Public License
%     along with this program.  If not, see <http://www.gnu.org/licenses/>.

\begin{frame}[t]{log}{過去の履歴を活用する}

  過去の履歴を閲覧
  \vspace{2ex}

  \onslide*<2>{
    \$ git log {\dhyphen}graph
    \vspace{2ex}

    \code{
      {\scriptsize
        *~~~commit~86ba5efc1bce4f65f517eb80a3cc5c21113bc637

        {\vbar}{\bslash}~~Merge:~ac24086~365850a

        {\vbar}~{\vbar}~Author:~Your~Name~{\textless}your@address{\textgreater}

        {\vbar}~{\vbar}~Date:~~~Wed~Jan~1~02:42:46~2014~+0900

        {\vbar}~{\vbar}

        {\vbar}~{\vbar}~~~~~Merge~branch~'develop'

        {\vbar}~{\vbar}

        {\vbar}~*~commit~365850a9b09bb6fbcfc3f762b630380053b46575

        {\vbar}~{\vbar}~Author:~Your~Name~{\textless}your@address{\textgreater}

        {\vbar}~{\vbar}~Date:~~~Wed~Jan~1~02:42:32~2014~+0900

        {\vbar}~{\vbar}

        {\vbar}~{\vbar}~~~~~Add~e.

        {\vbar}~{\vbar}

        ...
      }
    }
  }

  \onslide*<3>{git log は less を使用しているので、長い場合は j,k で上下移動、q で終了}

  \onslide*<4-6>{
    ---

    \code{
      commit~86ba5efc1bce4f65f517eb80a3cc5c21113bc637

      Merge:~ac24086~365850a

      Author:~Your~Name~{\textless}your@address{\textgreater}

      Date:~~~Wed~Jan~1~02:42:46~2014~+0900
    \vspace{2ex}

    Merge~branch~'develop'
    }

    ---
    \vspace{2ex}

    \onslide*<4>{これが一つの commit}
    \onslide*<5>{
      ``commit'' の横に書いてあるのが、その commit のハッシュ値

      (リンクでは無い、commit の本当の名前として使用)
    }
    \onslide*<6>{最後が commit message}
  }

\end{frame}


\begin{frame}[t]{git log コマンド補足}{過去の履歴を活用する}

  \begin{itemize}
  \item git log {\dhyphen}graph [\textit{commit}]

    で commit の先祖を表示

    commit は branch, HEAD, ハッシュ値等で指定

    commit のデフォルトは HEAD
    \vspace{2ex}

  \item ハッシュ値はその commit の本名の要な物

    git branch や git checkout で使用可能

    (自信が無い場合は、git checkout で commit のハッシュ値を指定する事はやめた方が無難)
    \vspace{2ex}

  \item git log で簡単に調べられるのは各 branch やその先祖

    例えば、現在の最新状態の branch を削除すると最新 commit を見つける事が困難になる

    (git branch -d でエラーが出るのはそのような場合)
  \end{itemize}

\end{frame}
