% This is part of the ``Git First Step''.
% Copyright 2014, 2020 Yoshida Shin.
% See the file slide.tex for copying conditions.

\begin{frame}[t]{checkout}{branch を使用する}

  HEAD の切り替えと、会わせて workspace と cached の変更
  \vspace{4ex}

  \onslide*<1>{
    workspace のファイル確認
    \vspace{2ex}

    \$ ls

    \code{c d}
  }

  \onslide*<2>{
    HEAD の切り替え実施
    \vspace{2ex}

    \$ git checkout develop
  }

  \onslide*<3>{
    workspace のファイル再確認
    \vspace{2ex}

    \$ ls

    \code{d}
  }

\end{frame}

\begin{frame}[t]{中で起きている事}{branch を使用する}

  \begin{columns}

    \begin{narrowcolumn}

      \begin{block}{workspace}
        \onslide<-5>{c} d
      \end{block}

      \begin{block}{cached (index)}
        \onslide<-5>{c} d
      \end{block}

    \end{narrowcolumn}

    \begin{halfcolumn}

      \begin{repository}{repository}
        \node (c3) [commit, rectangle split, rectangle split parts=3] at (0em, 10ex) {
          \commitmessage{Add c.}
          \nodepart{second}{c d}
          \nodepart{third}{master\onslide*<-3>{\HEAD}}
        };

        \node (c2) [commit, rectangle split, rectangle split parts=3] at (0em, 0ex) {
          \commitmessage{Rename a to d and delete b.}
          \nodepart{second}{d}
          \nodepart{third}{develop\onslide*<4->{\HEAD}}
        };

        \draw (c2) -- (c3);
      \end{repository}

    \end{halfcolumn}

  \end{columns}
  \vspace{2ex}

  \onslide*<2>{
    \$ ls
    \vspace{2ex}

    \code{c d}
  }

  \onslide*<3-6>{
    \$ git checkout develop
    \vspace{2ex}

    \onslide*<3-4>{HEAD のリンクの向き先を develop に変更}
    \onslide*<5-6>{あわせて、workspace や cached に新旧 HEAD の差分も適用}
  }

  \onslide*<7>{
    \$ ls
    \vspace{2ex}

    \code{d}
  }
\end{frame}


\begin{frame}[t]{git checkout コマンド補足}{branch を使用する}

  \begin{itemize}
  \item git checkout \textit{commit}

    で HEAD の commit へ切り替え

    workspace, cached も会わせて変更

    commit は branch, tag\footnote{後述}, ハッシュ値\footnote{後述}等で指定
    \vspace{2ex}

  \item リリース時以外は HEAD はどこかの branch を指している状態にすると良い
    \vspace{2ex}

  \item workspace や cached が clean でない場合\footnote{未 commit のファイルがある場合}は

    情報が失われる可能性があるので注意
  \end{itemize}

\end{frame}
