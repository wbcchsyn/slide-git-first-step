% This is part of the ``Git First Step''.
% Copyright 2014, 2020 Yoshida Shin.
% See the file slide.tex for copying conditions.

\begin{frame}[t]{revert}{その他}

  特定の commit を打ち消す commit を行う

  (commit とその親 commit の差分を打ち消す)
  \vspace{4ex}

  \onslide*<1>{
    戻したいバージョン (例えば、``Add h.'' という commit ) の commit のハッシュ値を確認
    \vspace{2ex}

    \$ git log {\dhyphen}graph
    \vspace{2ex}

    ...
    \vspace{2ex}

    (環境ごとに commit のハッシュ値は異なるので注意)

    今回は 06d957bc3322557eb76fc86ee87b66c0288d3c08 とする
  }


  \onslide*<2>{
    \$ git revert 06d9\footnote{``Add i.'' の commit}
  }

\end{frame}

\begin{frame}[t]{中で起きている事}{その他}

  \begin{columns}

    \begin{narrowcolumn}

      \begin{block}{workspace}
        c d e f g \onslide<-2>{h} i
      \end{block}

      \begin{block}{cached (index)}
        c d e f g \onslide<-2>{h} i
      \end{block}

    \end{narrowcolumn}

    \begin{halfcolumn}

      \begin{repository}{repository}

        \onslide<5->{
          \node (c10) [commit, rectangle split, rectangle split parts=1] at (0em, 12ex){
            \commitmessage{Revert ``Add h.''...}
          };
        }

        \node (c9) [commit, rectangle split, rectangle split parts=1] at (0em, 8ex){
          \commitmessage{Merge remote-...}
        };

        \node (c8) [commit, rectangle split, rectangle split parts=1] at (2em, 4ex){
          \commitmessage{Add i.}
        };

        \node (c7) [commit, rectangle split, rectangle split parts=1] at (-2em, 4ex){
          \commitmessage{Add h.}
        };

        \node (c6) [commit, rectangle split, rectangle split parts=1] at (0em, 0ex){
          \commitmessage{Add f.}
        };

        \onslide*<5->{\draw (c9) -- (c10);}
        \draw (c8) -- (c9);
        \draw (c7) -- (c9);
        \draw (c6) -- (c8);
        \draw (c6) -- (c7);

      \end{repository}

    \end{halfcolumn}

  \end{columns}
  \vspace{2ex}

  \onslide*<2-5>{
    \$ git revert 06d9
    \vspace{2ex}

    \onslide*<2-3>{
      06d9 (``Add h.'' の commit) の変更 (06d9 とその親 commit の差分)

      の逆を、workspace と cached に行う
    }

    \onslide*<4-5>{git commit を行う}
  }

\end{frame}


\begin{frame}[t]{git revert コマンド補足}{その他}

  \begin{itemize}
  \item git revert \textit{commit1} [\textit{commit2} [\textit{commit3} ...]]

    で特定の commit の変更を打ち消す commit を行う
    \vspace{2ex}

  \item 親が複数存在する commit\footnote{git merge で作成された commit} を指定するには、-m というオプションが必要

    (今回の範囲を超えるので詳細は割愛)
  \end{itemize}

\end{frame}
