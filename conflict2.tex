% This is part of the ``Git First Step''.
% Copyright 2014, 2020 Yoshida Shin.
% See the file slide.tex for copying conditions.

\begin{frame}[t]{衝突 (解決)}{更新が競合したら}

  衝突をその場で解決
  \vspace{4ex}

  \onslide*<1>{
    準備
    \vspace{2ex}

    \$ cd \~{}/local\_rep

    \$ git branch conflict3
  }

  \onslide*<2>{
    conflict3 で更新
    \vspace{2ex}

    \$ git checkout conflict3

    \$ echo ``conflict3'' {\textgreater}{\textgreater} f

    \$ git add f

    \$ git commit -m 'Update 3.'
  }

  \onslide*<3>{
    conflict3 で conflict1 と merge (衝突)
    \vspace{2ex}

    \$ git merge conflict1
  }

  \onslide*<4>{
    衝突しない様に編集、commit してから再度 merge
    \vspace{2ex}

    \$ vi f

    \$ git add f

    \$ git commit
  }
\end{frame}


\begin{frame}[t]{中で起きている事}{更新が競合したら}

  \begin{columns}

    \begin{narrowcolumn}

      \begin{block}{workspace}
        c d e
        \onslide*<1-5>{f}\onslide*<6-7>{\textcolor{red}{f}}\onslide*<8->{\textcolor{green}{f}}
        g h i
      \end{block}

      \begin{block}{cached (index)}
        c d e
        \onslide*<1-5>{f}\onslide*<6-7>{\textcolor{red}{f}}\onslide*<8->{\textcolor{green}{f}}
        g h i
      \end{block}

    \end{narrowcolumn}

    \begin{widecolumn}

      \begin{repository}{repository}
        \onslide<10->{
          \node (c15) [commit, rectangle split, rectangle split parts=2] at (3em, 10ex){
            \commitmessage{Merge bra...}
            \nodepart{second}{conflict3}
          };
        }

        \onslide*<10->{
          \node (c14) [commit, rectangle split, rectangle split parts=1] at (3em, 4ex){
            \commitmessage{Update 3.}
          };
        }

        \onslide*<4-9>{
          \node (c14) [commit, rectangle split, rectangle split parts=2] at (3em, 6ex){
            \commitmessage{Update 3.}
            \nodepart{second}{conflict3}
          };
        }

          \node (c10) [commit, rectangle split, rectangle split parts=2] at (-3em, 6ex){
            \commitmessage{Update 1.}
            \nodepart{second}{conflict1}
          };

        \node (c9) [commit, rectangle split, rectangle split parts=1] at (0em, 0ex){
          \commitmessage{Merge remote-tracking branch ...}
        };

        \onslide*<10->{\draw (c14) -- (c15);}
        \onslide*<10->{\draw (c10) -- (c15);}
        \onslide*<4->{\draw (c9) -- (c14);}
        \draw (c9) -- (c10);
      \end{repository}

    \end{widecolumn}

  \end{columns}
  \vspace{2ex}

  \onslide*<2>{
    \$ cd \~{}/local\_rep

    \$ git branch conflict3
  }

  \onslide*<3-4>{
    \$ git checkout conflict3

    \$ echo ``conflict3'' {\textgreater}{\textgreater} f

    \$ git add f

    \$ git commit -m 'Update3'
  }

  \onslide*<5-6>{
    \$ git merge conflict1

    workspace に無理矢理両方の更新を適用した状態
  }

  \onslide*<7-8>{
    \$ vi f

    \$ git add f
  }

  \onslide*<9-10>{\$ git commit}
\end{frame}


\begin{frame}[t]{conflict 補足}{更新が競合したら}

  \begin{itemize}
  \item git merge {\dhyphen}abort の過信は禁物

    workspace や cached が clean で無い場合\footnote{未 commit のファイルがある状態}、正しく戻らない可能性がある
    \vspace{2ex}

  \item conflict 解決時、勝手に他人の更新を変更する事は危険

    自分の更新を変更するか、競合する更新をした人に確認するべき
    \vspace{2ex}

  \item abort せずに直接 conflict を解消する際は、

    競合部分の削除と順番入れ替えのみで対応するべき

    (先祖の commit に含まれない更新を merge commit に入れるべきではない)

    (これ以外の手段で conflict を解消する方法は evil merge と言われている)
  \end{itemize}

\end{frame}
