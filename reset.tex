% This is part of the ``Git First Step''.
% Copyright 2014, 2020 Yoshida Shin.
% See the file slide.tex for copying conditions.

\begin{frame}[t]{reset}{過去の履歴を活用する}
  branch と HEAD を一度に切り替える

  (良く、過去のバージョンに戻すのに使用)
  \vspace{4ex}

  \onslide*<1>{
    戻したいバージョン (例えば、``Add e.'' という commit ) の commit のハッシュ値を確認
    \vspace{2ex}

    \$ git log {\dhyphen}graph
    \vspace{2ex}

    ...
    \vspace{2ex}

    (commit のハッシュ値 commit する度に異なるので注意)

    今回は 365850a9b09bb6fbcfc3f762b630380053b46575 とする
  }

  \onslide*<2>{
    準備
    \vspace{2ex}

    \$ git branch for\_soft

    \$ git branch for\_mixed

    \$ git branch for\_hard
  }

  \onslide*<3>{
    HEAD だけ古いバージョンに戻す
    \vspace{2ex}

    \$ git checkout for\_soft

    \$ git reset {\dhyphen}soft 3658\footnote{重複が無ければ、hash 値は先頭 4文字以上で大丈夫}
  }

  \onslide*<4>{
    HEAD と cached を古いバージョンに戻す
    \vspace{2ex}

    \$ git checkout for\_mixed

    \$ git reset {\dhyphen}mixed 3658
  }

  \onslide*<5>{
    HEAD, cached, workspace の全てを古いバージョンに戻す
    \vspace{2ex}

    \$ git checkout for\_hard

    \$ git reset {\dhyphen}hard 3658
  }

  \onslide*<6>{
    テスト環境後始末
    \vspace{2ex}

    \$ git checkout master

    \$ git branch -d for\_soft for\_mixed for\_hard
  }
\end{frame}

\begin{frame}[t]{中で起きている事}{過去の履歴を活用する}

  \begin{columns}

    \begin{narrowcolumn}

      \begin{block}{workspace}
        \onslide<1-13,15>{c}
        d e
      \end{block}

      \begin{block}{cached (index)}
        \onslide<1-9,12-13,15>{c}
        d e
      \end{block}

    \end{narrowcolumn}

    \begin{widecolumn}

      \begin{leftrepository}{repository}
        \node (c5) [commit, rectangle split, rectangle split parts=3] at (7em, 11ex){
          \commitmessage{Merge branch 'develop'}
          \nodepart{second}{c d e}
          \nodepart{third}{master\onslide*<-3,15>{\HEAD}\onslide*<2-5>{ for\_soft}\onslide*<4-5>{\HEAD}\onslide*<2-9>{ for\_mixed}\onslide*<8-9>{\HEAD}\onslide*<2-13>{ for\_hard}\onslide*<12-13>{\HEAD}}
        };

        \onslide<6-14>{
          \node (c4) [commit, rectangle split, rectangle split parts=3] at (9em, 0ex){
            \commitmessage{Add e.}
            \nodepart{second}{d e}
            \nodepart{third}{for\_soft\onslide*<-7>{\HEAD} \onslide*<10->{for\_mixed\onslide*<-11>{\HEAD}} \onslide*<14>{for\_hard{\HEAD}}}
          };
        }

        \onslide*<1-5,15>{
          \node (c4) [commit, rectangle split, rectangle split parts=2] at (9em, -1ex){
            \commitmessage{Add e.}
            \nodepart{second}{d e}
          };
        }

        \node (c3) [commit, rectangle split, rectangle split parts=1] at (0em, -2ex){
          \commitmessage{Add c.}
        };
        \draw (c4) -- (c5);
        \draw (c3) -- (c5);
      \end{leftrepository}

    \end{widecolumn}

  \end{columns}
  \vspace{2ex}

  \onslide*<2>{
    \$ git branch for\_soft

    \$ git branch for\_mixed

    \$ git branch for\_hard
  }

  \onslide*<3-4>{\$ git checkout for\_soft}
  \onslide*<5-6>{
    \$ git reset {\dhyphen}soft 3658
    \vspace{2ex}

    HEAD だけ変更する

    (HEAD が branch を指している時は branch を変更する)
  }

  \onslide*<7-8>{\$ git checkout for\_mixed}
  \onslide*<9-10>{
    \$ git reset {\dhyphen}mixed 3658
    \vspace{2ex}

    cached と HEAD を変更する
  }

  \onslide*<11-12>{\$ git checkout for\_hard}
  \onslide*<13-14>{
    \$ git reset {\dhyphen}hard 3658
    \vspace{2ex}

    workspace, cached, HEAD を変更する
  }

  \onslide*<15>{
    \$ git checkout master

    \$ git branch -d for\_soft for\_mixed for\_hard
  }
\end{frame}

\begin{frame}[t]{git reset コマンド補足}{過去の履歴を活用する}

  \begin{itemize}
  \item git reset [{\dhyphen}soft {\vbar} {\dhyphen}mixed {\vbar} {\dhyphen}hard] [\textit{comit}]

    で HEAD, branch の切り替えと (オプションによって) workspace や cached の変更を実施

    オプションのデフォルトは {\dhyphen}mixed、

    commit は branch, HEAD, ハッシュ値等で指定

    commit のデフォルトは HEAD
    \vspace{2ex}

  \item git reset とだけ実行すると、HEAD は変わらず cached が HEAD と同じ状態になる

    間違えて git add してしまった場合などに使用
    \vspace{2ex}

  \item HEAD を切り替えるので、git reset は git checkout と少し似ている

    違いは、branch も一緒に切り替えるかどうか
  \end{itemize}

\end{frame}
