% This is part of the ``Git First Step''.
% Copyright 2014, 2020 Yoshida Shin.
% See the file slide.tex for copying conditions.

\begin{frame}[t]{add, commit}{使ってみる}

  repository にファイル、ディレクトリを追加する
  \vspace{4ex}

  \onslide*<1>{
    workspace を更新
    \vspace{2ex}

    \$ touch a b c
  }

  \onslide*<2>{
    workspace を cached に反映
    \vspace{2ex}

    \$ git add a b
  }

  \onslide*<3>{
    cached を repository に反映
    \vspace{2ex}

    \$ git commit -m 'Add a and b.'
  }

\end{frame}


\begin{frame}[t]{中で起きている事}{使ってみる}

  \begin{columns}

    \begin{narrowcolumn}

      \begin{block}{workspace}
        \onslide<3->{a b c}
      \end{block}

      \begin{block}{cached (index)}
        \onslide<5->{a b}
      \end{block}

    \end{narrowcolumn}

    \begin{halfcolumn}

      \begin{repository}{repository}
        \onslide<8->{
          \node (c1) [commit, rectangle split, rectangle split parts=3] at (0em, 0ex) {
            \commitmessage{Add a and b.}
            \nodepart{second}{a b}
            \nodepart{third}{master\onslide*<9->{{\HEAD}}}
          };
        }

        \onslide*<7>{
          \node (c1) [commit, rectangle split, rectangle split parts=2] at (0em, 0ex) {
            \commitmessage{Add a and b.}
            \nodepart{second}{a b}
          };
        }
      \end{repository}

    \end{halfcolumn}

  \end{columns}
  \vspace{2ex}

  \onslide*<2-3>{
    \$ touch a b c
    \vspace{2ex}

    .git ディレクトリは触っていないので、workspace のみ変更
  }

  \onslide*<4-5>{
    \$ git add a b
    \vspace{2ex}

    \onslide*<-5>{workspace から cached へファイル、ディレクトリをコピー}
  }

  \onslide*<6->{
    \$ git commit -m 'Add a and b.'
    \vspace{2ex}

    \onslide*<-7>{cached を取り込み、message を添えて commit を作成}
    \onslide*<8>{
      また、初回 commit 時のみ master branchを作成

      (branch: commit へのリンクの様な物)
    }
    \onslide*<9>{
      最初、HEAD は master へ向いている

      (HEAD: repository 内の現在の commit を指すリンクの様な物)

      (この場合、HEAD はリンクのリンクとなる)
    }
  }

\end{frame}


\begin{frame}[t]{git add コマンド補足}{使ってみる}

  \begin{itemize}
  \item git add \textit{path1} [\textit{path2} [\textit{path3} ...]]\footnote{[ ] でくくってあるパラメータは省略可能という意味}

    で複数の path を一度に workspace へコピー可能
    \vspace{2ex}

  \item path が ディレクトリの場合、そのディレクトリ以下のファイルを再起的に git add する
    \vspace{2ex}

  \item workspace のトップディレクトリで git add . と行うと全ファイルを git add できる
  \end{itemize}

\end{frame}


\begin{frame}[t]{git commit コマンド補足}{}

  \begin{itemize}
  \item git commit

    でエディタが立ち上がり、そこで commit message を入力すると新 commit が作成される
    \vspace{2ex}

  \item git commit -m \textit{message}

    でエディタを立ち上げずに commit message を指定する事も可能
    \vspace{2ex}

  \item message の無い commit を作成する事はできない
  \end{itemize}

\end{frame}
