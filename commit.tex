% This is part of the ``Git First Step''.
% Copyright 2014, 2020 Yoshida Shin.
% See the file slide.tex for copying conditions.

\begin{frame}[t]{commit}{使ってみる}

  cached を repository に反映
  \vspace{2ex}

  \$ git commit -m 'Add a and b.'

\end{frame}


\begin{frame}[t]{中で起きている事}{使ってみる}

  \begin{columns}

    \begin{narrowcolumn}

      \begin{block}{workspace}
        a b c
      \end{block}

      \begin{block}{cached (index)}
        a b
      \end{block}

    \end{narrowcolumn}

    \begin{halfcolumn}

      \begin{repository}{repository}
        \onslide<3->{
          \node (c1) [commit, rectangle split, rectangle split parts=3] at (0em, 0ex) {
            \commitmessage{Add a and b.}
            \nodepart{second}{a b}
            \nodepart{third}{master\onslide*<4->{{\HEAD}}}
          };
        }

        \onslide*<2>{
          \node (c1) [commit, rectangle split, rectangle split parts=2] at (0em, 0ex) {
            \commitmessage{Add a and b.}
            \nodepart{second}{a b}
          };
        }
      \end{repository}

    \end{halfcolumn}

  \end{columns}
  \vspace{2ex}

  \$ git commit -m 'Add a and b.'
  \vspace{2ex}

  \onslide*<-2>{cached を取り込み、message を添えて commit を作成}
  \onslide*<3>{
    また、初回 commit 時のみ master branchを作成

    (branch: commit へのリンクの様な物)
  }
  \onslide*<4>{
    最初、HEAD は master へ向いている

    (HEAD: repository 内の現在の commit を指すリンクの様な物)

    (この場合、HEAD はリンクのリンクとなる)
  }

\end{frame}


\begin{frame}[t]{git commit コマンド補足}{}

  \begin{itemize}
  \item git commit

    でエディタが立ち上がり、そこで commit message を入力すると新 commit が作成される
    \vspace{2ex}

  \item git commit -m \textit{message}

    でエディタを立ち上げずに commit message を指定する事も可能
    \vspace{2ex}

  \item message の無い commit を作成する事はできない
  \end{itemize}

\end{frame}
