% This is part of the ``Git First Step''.
% Copyright 2014, 2020 Yoshida Shin.
% See the file slide.tex for copying conditions.

\begin{frame}<1-3>[t]{clone}{複数人で開発する}
  repository のコピーを作成
  \vspace{2ex}

  \onslide*<1>{
    clone repository の作成
    \vspace{2ex}

    \$ git clone \~{}/remote\_rep \~{}/clone\_rep

    \$ cd \~{}/clone\_rep
  }

  \onslide*<2>{
    現在の repository 一覧を表示
    \vspace{2ex}

    \$ git branch -a

    * master

    \onslide<99>{* }remotes/origin/HEAD -\textgreater origin/master

    \onslide<99>{* }remotes/origin/master

    \onslide<99>{* }remotes/origin/develop
  }

  \onslide*<3>{
    develop branch の作成
    \vspace{2ex}

    \$ git branch develop remotes/origin/develop
  }
\end{frame}

\begin{frame}[t]{中で起きている事}{複数人で開発する}

  \begin{columns}

    \begin{halfcolumn}

      \begin{repository}{remote}
        \node (c6) [commit, rectangle split, rectangle split parts=2] at (0em, 11ex){
          \commitmessage{Add f.}
          \nodepart{second}{master{\HEAD}}
        };

        \node (c5) [commit, rectangle split, rectangle split parts=2] at (0em, -1ex){
          \commitmessage{Merge branch 'develop'}
          \nodepart{second}{develop}
        };

        \draw (c5) -- (c6);
      \end{repository}

    \end{halfcolumn}

    \begin{column}{0.55\textwidth}

      \begin{repository}{clone}
        \onslide<5->{
          \node (c6) [commit, rectangle split, rectangle split parts=3] at (0em, 10ex){
            \commitmessage{Add f.}
            \nodepart{second}{master{\HEAD}}
            \nodepart{third}{\remotebranch{remotes/origin/master remotes/origin/HEAD}}
          };
        }

        \onslide*<3-4>{
          \node (c6) [commit, rectangle split, rectangle split parts=2] at (0em, 10ex){
            \commitmessage{Add f.}
            \nodepart{second}{\remotebranch{remotes/origin/master remotes/origin/HEAD}}
          };
        }

        \onslide<8->{
          \node (c5) [commit, rectangle split, rectangle split parts=3] at (0em, 0ex){
            \commitmessage{Merge branch 'develop'}
            \nodepart{second}{develop}
            \nodepart{third}{\remotebranch{remotes/origin/develop}}
          };
        }

        \onslide*<3-7>{
          \node (c5) [commit, rectangle split, rectangle split parts=2] at (0em, 0ex){
            \commitmessage{Merge branch 'develop'}
            \nodepart{second}{\remotebranch{remotes/origin/develop}}
          };
        }

        \onslide*<3->{\draw (c5) -- (c6);}
      \end{repository}

    \end{column}

  \end{columns}
  \vspace{2ex}

  \onslide*<1-5>{
    \$ git clone \~{}/remote\_rep \~{}/clone\_rep

    \$ cd \~{}/clone\_rep
    \vspace{2ex}

  }

  \onslide*<2-3>{
    remote\_rep の全 branch, HEAD を

    remote branch としてコピー
  }

  \onslide*<4-5>{remotes/origin/HEAD を元に master branch を作成}

  \onslide*<6>{
    \$ git branch -a

    \code{
      * master

      \onslide<99>{* }remotes/origin/HEAD -\textgreater origin/master

      \onslide<99>{* }remotes/origin/develop

      \onslide<99>{* }remotes/origin/master
    }
  }

  \onslide*<7-8>{
    \$ git branch develop remotes/origin/develop
    \vspace{2ex}

    remotes/origin/develop を元に develop branch を作成
  }

\end{frame}


\begin{frame}[t]{git clone コマンド補足}{複数人で開発する}

  \begin{itemize}
  \item git clone \textit{URI} [\textit{directory}]

    で remote repository の clone を directory に作成

    directory のデフォルトは URI の reopsitory 名

    git の top directory はいつでも変更してよい
    \vspace{2ex}

  \item git clone 時に指定した URI は、origin というエイリアスで登録される
    \vspace{2ex}

  \item git branch -a

    で remote branch を含む全ての branch が閲覧可能
    \vspace{2ex}

  \item master 以外の必要な branch は

    git branch \textit{branch-name} \textit{start-point} で作成

    (remote branch は閲覧専用として使用するべき)
  \end{itemize}

\end{frame}
