% This is part of the ``Git First Step''.
% Copyright 2014, 2020 Yoshida Shin.
% See the file slide.tex for copying conditions.

\begin{frame}[t]{最後に}{}

  もっと git を勉強するには
  \vspace{2ex}

  \begin{itemize}
  \item man を読む
  \item 公式ガイドを読む
  \item 一般的な git の開発フローを学ぶ
  \end{itemize}
  \vspace{2ex}

  などの方法がおすすめ

\end{frame}

\begin{frame}[t]{最後に}{}

  man の探し方
  \vspace{2ex}

  git の man は、``git'' と最初の引数を ``-'' でつなげた物を読む

  例) \$ man git-add
  \vspace{2ex}

  \onslide*<2->{
    最初の引数まで入力して {\dhyphen}help と入力しても同じ

    例) \$ git add {\dhyphen}help
  }
  \vspace{2ex}

  \onslide*<3>{
    そもそも最初の引数が分からない場合や、

    git でどんな事が出来るか知りたい場合は

    git の man を引く

    \$ man git
  }
\end{frame}


\begin{frame}[t]{最後に}{}

  \onslide*<1>{
    公式ガイドの見方

    \$ git help -g

    で git のガイド一覧を表示
  }

  \onslide*<2>{
    \$ git help -g

    \code{
      The common Git guides are:
      \vspace{2ex}

      attributes   Defining attributes per path

      glossary     A Git glossary

      ignore       Specifies intentionally untracked files to ignore

      modules      Defining submodule properties

      revisions    Specifying revisions and ranges for Git

      tutorial     A tutorial introduction to Git (for version 1.5.1 or newer)

      workflows    An overview of recommended workflows with Git
      \vspace{2ex}

      'git help -a' and 'git help -g' lists available subcommands and some

      concept guides. See 'git help {\textless}command{\textgreater}' or 'git help {\textless}concept{\textgreater}'

      to read about a specific subcommand or concept.
    }
  }

  \onslide*<3>{
    興味があるガイドをみつけたら
    \vspace{2ex}

    \$ git help {\textless}concept{\textgreater}
    \vspace{2ex}

    で閲覧
    \vspace{2ex}

    例) git help tutorial
  }
\end{frame}


\begin{frame}[t]{最後に}{}
  一般的な開発フローについて学ぶ
  \vspace{2ex}

  本勉強会ではあまり触れませんでしたが

  best practice についても勉強すると面白いかも
  \vspace{2ex}

  まずは、git flow と github flow を勉強すると良いと思います
\end{frame}


\begin{frame}[t]{最後に}{}

  ``git 次の一歩'' も開催予定なので、

  興味のある方は来てください

\end{frame}
