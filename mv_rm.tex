% This is part of the ``Git First Step''.
% Copyright 2014, 2020 Yoshida Shin.
% See the file slide.tex for copying conditions.

\begin{frame}[t]{mv, rm}{概要}
  cached からファイルを削除するには workspace から削除した後で git add すれば良い
  \vspace{2ex}

  \onslide<2->{
    ファイルのリネームはファイル削除と追加を行えば良い
  }
  \vspace{2ex}


  \onslide<3->{
    ただし、この方法は面倒でタブ補完が効かないこともある
  }
  \vspace{2ex}

  \onslide<4->{
    ショートカット的なコマンドとして、``git rm'' と ``git mv'' が用意されている
  }
\end{frame}


\begin{frame}[t]{mv, rm}{概要}
  \$ git rm [\hyphen r] [\hyphen f] \textit{path1} [\textit{path2} [\textit{path3} [...]]]
  \vspace{2ex}

  これは、
  \vspace{2ex}

  \$ rm [\hyphen r] [\hyphen f] \textit{path1} [\textit{path2} [\textit{path3} [...]]]

  \$ git add \textit{path1} [\textit{path2} [\textit{path3} [...]]]
  \vspace{2ex}

  とほとんど同じ
  \vspace{2ex}

  \onslide<2->{
    git rm では git の使い方に対応したエラーを出してくれるところが少し違う
    \vspace{2ex}

    \onslide<3->{
      エラーメッセージを見て問題無いと思ったら \hyphen f をつけて再実行すると良い

      (個人的には少し過剰警告だと思う)
    }
  }
\end{frame}


\begin{frame}[t]{mv, rm}{概要}
  \$ git mv \textit{path1} \textit{path2}
  \vspace{2ex}

  これは、
  \vspace{2ex}

  \$ mv \textit{path1} \textit{path2}

  \$ git add \textit{path1} \textit{path2}
  \vspace{2ex}

  とほとんど同じ
\end{frame}
