% This is part of the ``Git First Step''.
% Copyright 2014, 2020 Yoshida Shin.
% See the file slide.tex for copying conditions.

\begin{frame}[t]{mv, rm}{使ってみる}

  repository からファイルの削除、移動 (リネーム)
  \vspace{4ex}

  \onslide*<1>{
    workspace, cached でリネーム
    \vspace{2ex}

    \$ git mv a d
  }

  \onslide*<2>{
    workspace, cached で削除
    \vspace{2ex}

    \$ git rm b
  }

  \onslide*<3>{
    cached を repository に反映
    \vspace{2ex}

    \$ git commit -m 'Rename a to d and delete b.'
  }

\end{frame}


\begin{frame}<1-10>[t]{中で起きている事}{使ってみる}

  \begin{columns}

    \begin{narrowcolumn}

      \begin{block}{workspace}
        \onslide<-2>{a}
        \onslide<-4>{b}
        c
        \onslide<3->{d}
      \end{block}

      \begin{block}{cached (index)}
        \onslide<-2>{a}
        \onslide<-4>{b}
        \onslide<99>{c}
        \onslide<3->{d}
      \end{block}

    \end{narrowcolumn}

    \begin{halfcolumn}

      \begin{repository}{repository}
        \onslide<9->{
          \node (c2) [commit, rectangle split, rectangle split parts=3] at (0em, 10ex) {
            \commitmessage{Rename a to d and delete b.}
            \nodepart{second}{d}
            \nodepart{third}{master{\HEAD}}
          };
        }

        \onslide*<7-8>{
          \node (c2) [commit, rectangle split, rectangle split parts=2] at (0em, 10ex) {
            \commitmessage{Rename a to d and delete b.}
            \nodepart{second}{d}
          };
        }

        \onslide<-8>{
          \node (c1) [commit, rectangle split, rectangle split parts=3] at (0em, 0ex) {
            \commitmessage{Add a and b.}
            \nodepart{second}{a b}
            \nodepart{third}{master{\HEAD}}
          };
        }

        \onslide*<9->{
          \node (c1) [commit, rectangle split, rectangle split parts=2] at (0em, -1.0ex) {
            \commitmessage{Add a and b.}
            \nodepart{second}{a b}
          };
        }

        \onslide<7->{\draw (c1) -- (c2);}
      \end{repository}

    \end{halfcolumn}

  \end{columns}
  \vspace{2ex}

  \onslide*<2-3>{
    \$ git mv a d
    \vspace{2ex}

    workspace, cached の両方でファイルを移動 (リネーム)
  }

  \onslide*<4-5>{
    \$ git rm b
    \vspace{2ex}

    workspace, cached の両方からファイルを削除
  }

  \onslide*<6-10>{
    \$ git commit -m 'Rename a to d and delete b.'
    \vspace{2ex}

    \onslide*<6-7>{
      cached を取り込み、message を添えて、現在の HEAD を親とする commit を作成
    }
    \onslide*<8-9>{
      HEAD が branch を指している時 (普通はこの状態) は、

      その branch の向き先を新しい commit へ変更
    }
    \onslide*<10>{
      HEAD は master へ向いているので、

      結果的に新しい commit へ移動}
  }

\end{frame}


\begin{frame}[t]{git mv コマンド補足}{使ってみる}

  \begin{itemize}
  \item git mv [-f] \textit{source} \textit{destination}

    で source を destination に移動
    \vspace{2ex}

  \item 通常の mv コマンドのように、destination がディレクトリの場合、source は destination 以下のディレクトリに移動
    \vspace{2ex}

  \item -f をつけると、既存のファイルを上書き出来る

    (-f が無い場合、エラーとなる)
  \end{itemize}

\end{frame}


\begin{frame}[t]{git rm コマンド補足}{使ってみる}

  \begin{itemize}
  \item git rm [-r] [-f] \textit{path1} [\textit{path2} [\textit{path3} ...]]

    で複数の path を削除可能
    \vspace{2ex}

  \item 通常の rm コマンドのように、ディレクトリを削除する場合は -r オプションが必要
    \vspace{2ex}

  \item workspace と cached の内容が違う時などは、誤操作防止のために git rm はエラーになる

    そのような場合、-f を使うと正常に削除可能
  \end{itemize}

\end{frame}
