% This is part of the ``Git First Step''.
% Copyright 2014, 2020 Yoshida Shin.
% See the file slide.tex for copying conditions.

\begin{frame}[t]{git の管理するファイル情報}{使ってみる}

  git が管理するファイル情報は、主に以下の 3 個

  \begin{itemize}
  \item ファイルの相対パス
  \item ファイルの中身
  \item ファイルのパーミッション
  \end{itemize}
  \vspace{2ex}

  \onslide<2->{
    逆に、git は以下のような情報は管理しません。
    \begin{itemize}
    \item ファイルの所有者
    \item ファイルの更新時刻
    \end{itemize}
  }
  \vspace{2ex}

  \onslide<3->{
    上に列挙した git が管理するファイル情報は ``git add'' で
    何度でも workspace から cached に反映可能
  }

\end{frame}


\begin{frame}[t]{パーミッションの変更}{使ってみる}

  a のパーミッションを変更する
  \vspace{4ex}

  \onslide*<1>{
    workspace を更新
    \vspace{2ex}

    \$ chmod +x b
  }

  \onslide*<2>{
    workspace を cached に反映
    \vspace{2ex}

    \$ git add b
  }

\end{frame}


\begin{frame}[t]{中で起きている事}{使ってみる}

  \begin{columns}

    \begin{narrowcolumn}

      \begin{block}{workspace}
        a \onslide*<1-2>{b \color{black}}\onslide*<3->{\color{green} b \color{black}}c
      \end{block}

      \begin{block}{cached (index)}
        a \onslide*<-4>{b}\onslide*<5->{\color{green}b}
      \end{block}

    \end{narrowcolumn}

    \begin{halfcolumn}

      \begin{repository}{repository}
        \node (c1) [commit, rectangle split, rectangle split parts=3] at (0em, 0ex) {
          \commitmessage{Add a and b.}
          \nodepart{second}{a b}
          \nodepart{third}{master{\HEAD}}
        };
      \end{repository}

    \end{halfcolumn}

  \end{columns}
  \vspace{2ex}

  \onslide*<2-3>{
    \$ chmod +x b
    \vspace{2ex}

    .git ディレクトリは触っていないので、workspace のみ変更
  }

  \onslide*<4-5>{
    \$ git add b
    \vspace{2ex}

    \onslide*<-5>{workspace から cached へファイル、ディレクトリをコピー}
  }
\end{frame}


\begin{frame}[t]{ファイルの中身を変更}{使ってみる}

  b の中身を変更する
  \vspace{4ex}

  \onslide*<1>{
    workspace を更新
    \vspace{2ex}

    \$ echo 'foo' \textgreater \space b
  }

  \onslide*<2>{
    workspace を cached に反映
    \vspace{2ex}

    \$ git add b
  }

\end{frame}


\begin{frame}[t]{中で起きている事}{使ってみる}

  \begin{columns}

    \begin{narrowcolumn}

      \begin{block}{workspace}
        a \onslide*<1-2>{\color{green}b \color{black}}\onslide*<3->{\color{red}b \color{black}}c
      \end{block}

      \begin{block}{cached (index)}
        a \onslide*<-4>{\color{green}b \color{black}}\onslide*<5->{\color{red}b \color{black}}
      \end{block}

    \end{narrowcolumn}

    \begin{halfcolumn}

      \begin{repository}{repository}
        \node (c1) [commit, rectangle split, rectangle split parts=3] at (0em, 0ex) {
          \commitmessage{Add a and b.}
          \nodepart{second}{a b}
          \nodepart{third}{master{\HEAD}}
        };
      \end{repository}

    \end{halfcolumn}

  \end{columns}
  \vspace{2ex}

  \onslide*<2-3>{
    \$ echo 'foo' \textgreater \space b
    \vspace{2ex}

    .git ディレクトリは触っていないので、workspace のみ変更
  }

  \onslide*<4-5>{
    \$ git add b
    \vspace{2ex}

    \onslide*<-5>{workspace から cached へファイル、ディレクトリをコピー}
  }
\end{frame}


\begin{frame}[t]{ファイルの削除}{使ってみる}

  b を削除する
  \vspace{4ex}

  \onslide*<1>{
    workspace を更新
    \vspace{2ex}

    \$ rm b
  }

  \onslide*<2>{
    workspace を cached に反映
    \vspace{2ex}

    \$ git add b
  }

\end{frame}


\begin{frame}[t]{中で起きている事}{使ってみる}

  \begin{columns}

    \begin{narrowcolumn}

      \begin{block}{workspace}
        a \onslide<1-2>{\color{red}b \color{black}}c
      \end{block}

      \begin{block}{cached (index)}
        a \onslide<-4>{\color{red}b \color{black}}
      \end{block}

    \end{narrowcolumn}

    \begin{halfcolumn}

      \begin{repository}{repository}
        \node (c1) [commit, rectangle split, rectangle split parts=3] at (0em, 0ex) {
          \commitmessage{Add a and b.}
          \nodepart{second}{a b}
          \nodepart{third}{master{\HEAD}}
        };
      \end{repository}

    \end{halfcolumn}

  \end{columns}
  \vspace{2ex}

  \onslide*<2-3>{
    \$ rm b
    \vspace{2ex}

    .git ディレクトリは触っていないので、workspace のみ変更
  }

  \onslide*<4-5>{
    \$ git add b
    \vspace{2ex}

    \onslide*<-5>{workspace から cached へファイル、ディレクトリをコピー

      (「存在しない」という状態をコピーしたので削除)
    }
  }
\end{frame}


\begin{frame}[t]{ファイルのリネーム}{使ってみる}

  a を d というファイル名にリネームする

  (ファイル追加と削除を行えばリネームになる)
  \vspace{4ex}

  \onslide*<1>{
    workspace を更新
    \vspace{2ex}

    \$ mv a d
  }

  \onslide*<2>{
    workspace を cached に反映
    \vspace{2ex}

    \$ git add a d
  }

\end{frame}


\begin{frame}[t]{中で起きている事}{使ってみる}

  \begin{columns}

    \begin{narrowcolumn}

      \begin{block}{workspace}
        \onslide<-2>{a }c \onslide<3->{d}
      \end{block}

      \begin{block}{cached (index)}
        \onslide<-4>{a }\onslide<5->{d}
      \end{block}

    \end{narrowcolumn}

    \begin{halfcolumn}

      \begin{repository}{repository}
        \node (c1) [commit, rectangle split, rectangle split parts=3] at (0em, 0ex) {
          \commitmessage{Add a and b.}
          \nodepart{second}{a b}
          \nodepart{third}{master{\HEAD}}
        };
      \end{repository}

    \end{halfcolumn}

  \end{columns}
  \vspace{2ex}

  \onslide*<2-3>{
    \$ mv a d
    \vspace{2ex}

    .git ディレクトリは触っていないので、workspace のみ変更
  }

  \onslide*<4-5>{
    \$ git add a d
    \vspace{2ex}

    \onslide*<-5>{workspace から cached へファイル、ディレクトリをコピー}
  }
\end{frame}


\begin{frame}[t]{git add コマンドまとめ}{使ってみる}

  \begin{itemize}
  \item git add \textit{path1} [\textit{path2} [\textit{path3} [...]]]

    で workspace の指定された path を cached にコピー
    \vspace{2ex}

  \item path が ディレクトリの場合、そのディレクトリ以下のファイルを再起的に git add する
    \vspace{2ex}

  \item workspace のトップディレクトリで git add . と行うと全ファイルを git add できる
  \item git add は何度でも実行可能

    (同じファイルを複数回指定しても良い)
  \item ファイルのパーミッションも git add で変更可能
  \item 存在しないファイルを git add すると削除になる
  \end{itemize}

\end{frame}


\begin{frame}[t]{commit の追加}{使ってみる}

  cached を repository に反映
  \vspace{2ex}

  \$ git commit -m 'Rename a to d and delete b.'
\end{frame}


\begin{frame}[t]{中で起きている事}{使ってみる}

  \begin{columns}

    \begin{narrowcolumn}

      \begin{block}{workspace}
        c d
      \end{block}

      \begin{block}{cached (index)}
        d
      \end{block}

    \end{narrowcolumn}

    \begin{halfcolumn}

      \begin{repository}{repository}
        \onslide<5->{
          \node (c2) [commit, rectangle split, rectangle split parts=3] at (0em, 10ex) {
            \commitmessage{Rename a to d and delete b.}
            \nodepart{second}{d}
            \nodepart{third}{master{\HEAD}}
          };
        }

        \onslide*<3-4>{
          \node (c2) [commit, rectangle split, rectangle split parts=2] at (0em, 10ex) {
            \commitmessage{Rename a to d and delete b.}
            \nodepart{second}{d}
          };
        }

        \onslide<-4>{
          \node (c1) [commit, rectangle split, rectangle split parts=3] at (0em, 0ex) {
            \commitmessage{Add a and b.}
            \nodepart{second}{a b}
            \nodepart{third}{master{\HEAD}}
          };
        }

        \onslide*<5->{
          \node (c1) [commit, rectangle split, rectangle split parts=2] at (0em, -1.0ex) {
            \commitmessage{Add a and b.}
            \nodepart{second}{a b}
          };
        }

        \onslide<3->{\draw (c1) -- (c2);}
      \end{repository}

    \end{halfcolumn}

  \end{columns}
  \vspace{2ex}

  \onslide<2->{
    \$ git commit -m 'Rename a to d and delete b.'
    \vspace{2ex}

    \onslide*<2-3>{
      cached を取り込み、message を添えて現在の HEAD を親とする commit を作成
    }
    \onslide*<4-5>{
      HEAD が branch を指している時 (普通はこの状態) は、

      その branch の向き先を新しい commit へ変更
    }
    \onslide*<6>{
      HEAD は master へ向いているので、

      結果的に新しい commit へ移動
    }
  }
\end{frame}
