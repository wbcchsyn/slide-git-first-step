% This is part of the ``Git First Step''.
% Copyright 2014, 2020 Yoshida Shin.
% See the file slide.tex for copying conditions.

\begin{frame}[t]{git の管理するファイル情報}{使ってみる}

  git が管理するファイル情報は、主に以下の 3 個

  \begin{itemize}
  \item ファイルの相対パス
  \item ファイルの中身
  \item ファイルのパーミッション
  \end{itemize}
  \vspace{2ex}

  \onslide<2->{
    逆に、git は以下のような情報は管理しません。
    \begin{itemize}
    \item ファイルの所有者
    \item ファイルの更新時刻
    \end{itemize}
  }
  \vspace{2ex}

  \onslide<3->{
    上に列挙した git が管理するファイル情報は ``git add'' で
    何度でも workspace から cached に反映可能
  }

\end{frame}


\begin{frame}[t]{パーミッションの変更}{使ってみる}

  a のパーミッションを変更する
  \vspace{4ex}

  \onslide*<1>{
    workspace を更新
    \vspace{2ex}

    \$ chmod +x b
  }

  \onslide*<2>{
    workspace を cached に反映
    \vspace{2ex}

    \$ git add b
  }

\end{frame}


\begin{frame}[t]{中で起きている事}{使ってみる}

  \begin{columns}

    \begin{narrowcolumn}

      \begin{block}{workspace}
        a \onslide*<1-2>{b \color{black}}\onslide*<3->{\color{green} b \color{black}}c
      \end{block}

      \begin{block}{cached (index)}
        a \onslide*<-4>{b}\onslide*<5->{\color{green}b}
      \end{block}

    \end{narrowcolumn}

    \begin{halfcolumn}

      \begin{repository}{repository}
        \node (c1) [commit, rectangle split, rectangle split parts=3] at (0em, 0ex) {
          \commitmessage{Add a and b.}
          \nodepart{second}{a b}
          \nodepart{third}{master{\HEAD}}
        };
      \end{repository}

    \end{halfcolumn}

  \end{columns}
  \vspace{2ex}

  \onslide*<2-3>{
    \$ chmod +x b
    \vspace{2ex}

    .git ディレクトリは触っていないので、workspace のみ変更
  }

  \onslide*<4-5>{
    \$ git add b
    \vspace{2ex}

    \onslide*<-5>{workspace から cached へファイル、ディレクトリをコピー}
  }
\end{frame}
