% This is part of the ``Git First Step''.
% Copyright 2014, 2020 Yoshida Shin.
% See the file slide.tex for copying conditions.

\begin{frame}[t]{merge}{branch を使用する}

  他のブランチの更新を取り込む
  \vspace{4ex}

  \onslide*<1>{
    develop を更新
    \vspace{2ex}

    \$ touch e

    \$ git add e

    \$ git commit -m 'Add e.'
  }

  \onslide*<2>{
    master に develop の更新を反映
    \vspace{2ex}

    \$ git checkout master

    \$ git merge develop
  }

\end{frame}

\begin{frame}[t]{中で起きている事}{branch を使用する}

  \begin{columns}

    \begin{narrowcolumn}

      \begin{block}{workspace}
        \onslide<5->{c} d \onslide<3,4,8->{e}
      \end{block}

      \begin{block}{cached (index)}
        \onslide<5->{c} d \onslide<3,4,10->{e}
      \end{block}

    \end{narrowcolumn}

    \begin{widecolumn}

      \begin{leftrepository}{repository}
        \onslide<12->{
          \node (c5) [commit, rectangle split, rectangle split parts=2] at (0em, 15ex){
            \commitmessage{Merge branch 'develop'}
            \nodepart{second}{master{\HEAD}}
          };
        }

        \onslide<3->{
          \node (c4) [commit, rectangle split, rectangle split parts=2] at (5em, 7ex){
            \commitmessage{Add e.}
            \nodepart{second}{develop\onslide*<-4>{{\HEAD}}}
          };
        }

        \onslide*<12>{
          \node (c3) [commit, rectangle split, rectangle split parts=1] at (0em, 6.5ex){
            \commitmessage{Add c.}
          };
        }

        \onslide*<-11>{
          \node (c3) [commit, rectangle split, rectangle split parts=2] at (0em, 7ex){
            \commitmessage{Add c.}
            \nodepart{second}{master\onslide*<5->{\HEAD}}
          };
        }

        \onslide<-2>{
          \node (c2) [commit, rectangle split, rectangle split parts=2] at (0em, 0ex){
            \commitmessage{Rename a to d and delete b.}
            \nodepart{second}{develop{\HEAD}}
          };
        }

        \onslide*<3->{
          \node (c2) [commit, rectangle split, rectangle split parts=1] at (0em, -1ex){
            \commitmessage{Rename a to d and delete b.}
          };
        }

        \onslide*<12->{\draw (c4) -- (c5);}
        \onslide*<12->{\draw (c3) -- (c5);}
        \onslide*<3->{\draw (c2) -- (c4);}
        \draw (c2) -- (c3);
      \end{leftrepository}

    \end{widecolumn}

  \end{columns}
  \vspace{2ex}

  \onslide*<2-3>{
    \$ touch e

    \$ git add e

    \$ git commit -m 'Add e.'
  }

  \onslide*<4-5>{
    \$ git checkout master
  }

  \onslide*<6-12>{
    \$ git merge develop
    \vspace{2ex}

    \onslide*<6>{
      HEAD と develop の共通の先祖の commit を探す

      ``Rename a to d and delete b.'' という commit が相当
    }

    \onslide*<7-8>{
      branch が分岐してから develop で行われた

      全変更を workspace に適用
    }
    \onslide*<9-10>{cached に適用}
    \onslide*<11-12>{commit の作成}
  }
\end{frame}


\begin{frame}[t]{git merge コマンド補足}{branch を使用する}

  \begin{itemize}
  \item 実際には branch の状態によって merge 後の

    repository の状態は異なる

    (ここで挙げた挙動は一例)
    \vspace{2ex}

  \item コマンド実施後に message 入力画面が出てくる事と

    出てこない事があるが

    今回の範囲を超えるので詳細は割愛
    \vspace{2ex}

  \item いずれの場合でも、最新状態では 2個の branch の

    変更は全て含まれる
  \end{itemize}

\end{frame}
