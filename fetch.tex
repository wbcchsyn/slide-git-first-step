% This is part of the ``Git First Step''.
% Copyright 2014, 2020 Yoshida Shin.
% See the file slide.tex for copying conditions.

\begin{frame}[t]{fetch}{複数人で開発する}

  remote repository の更新を反映
  \vspace{4ex}

  \onslide*<1>{
    local\_rep で更新
    \vspace{2ex}

    \$ cd \~{}/local\_rep

    \$ git add g

    \$ git commit -m 'Add g.'

    \$ git push origin master
  }

  \onslide*<2>{
    clone\_rep に反映
    \vspace{2ex}

    \$ cd \~{}/clone\_rep

    \$ git fetch {\dhyphen}all -p

    \$ git merge remotes/origin/master
  }

\end{frame}

\begin{frame}[t]{中で起きている事}{複数人で開発する}

  \begin{columns}

    \begin{narrowcolumn}

      \begin{repository}{local}
        \onslide<2->{
          \node (c7) [commit, rectangle split, rectangle split parts=1] at (0em, 6.5ex){
            \commitmessage{Add g.}
          };
        }

        \node (c6) [commit, rectangle split, rectangle split parts=1] at (0em, -2ex){
          \commitmessage{Add f.}
        };

        \onslide*<2->{\draw (c6) -- (c7);}
      \end{repository}

    \end{narrowcolumn}

    \begin{narrowcolumn}

      \begin{repository}{remote}
        \onslide<3->{
          \node (c7) [commit, rectangle split, rectangle split parts=1] at (0em, 6.5ex){
            \commitmessage{Add g.}
          };
        }

        \node (c6) [commit, rectangle split, rectangle split parts=1] at (0em, -2ex){
          \commitmessage{Add f.}
        };

        \onslide*<3->{\draw (c6) -- (c7);}
      \end{repository}

    \end{narrowcolumn}

    \begin{halfcolumn}

      \begin{repository}{clone}

        \onslide<8>{
          \node (c7) [commit, rectangle split, rectangle split parts=3] at (0em, 5ex){
            \commitmessage{Add g.}
            \nodepart{second}{master{\HEAD}}
            \nodepart{third}{\remotebranch{remotes/origin/master}}
          };
        }

        \onslide*<6-7>{
          \node (c7) [commit, rectangle split, rectangle split parts=2] at (0em, 6ex){
            \commitmessage{Add g.}
            \nodepart{second}{\remotebranch{remotes/origin/master}}
          };
        }

        \onslide<8>{
          \node (c6) [commit, rectangle split, rectangle split parts=1] at (0em, -3ex){
            \commitmessage{Add f.}
          };
        }

        \onslide*<6-7>{
          \node (c6) [commit, rectangle split, rectangle split parts=2] at (0em, -1.5ex){
            \commitmessage{Add f.}
            \nodepart{second}{master{\HEAD}}
          };
        }

        \onslide*<-5>{
          \node (c6) [commit, rectangle split, rectangle split parts=3] at (0em, 0ex){
            \commitmessage{Add f.}
            \nodepart{second}{master{\HEAD}}
            \nodepart{third}{\remotebranch{remotes/origin/master}}
          };
        }

        \onslide*<6->{\draw (c6) -- (c7);}
      \end{repository}

    \end{halfcolumn}

  \end{columns}
  \vspace{2ex}

  \onslide*<2>{
    \$ cd \~{}/local\_rep

    \$ git add g

    \$ git commit -m 'Add g.'
  }

  \onslide*<3>{\$ git push origin master}

  \onslide*<4>{\$ cd \~{}/clone\_rep}

  \onslide*<5-6>{
    \$ git fetch {\dhyphen}all -p
    \vspace{2ex}

    remotes 以下の branch を最新版にバージョンアップ
  }

  \onslide*<7-8>{
    \$ git merge remotes/origin/master
    \vspace{2ex}

    master に remotes/origin/master の更新を適用
  }
\end{frame}


\begin{frame}[t]{fetch コマンド補足}{複数人で開発する}

  git fetch [{\dhyphen}all] [-p]

  \begin{itemize}
  \item {\dhyphen}all は全ての remotes 以下の branch を最新にする

    (local に存在しない branch は新規作成する)
  \item -p は remote reopsitory から削除された branch を local からも削除する
  \item git pull のデフォルトの動作は、git fetch と git merge を行うもの
  \end{itemize}
\end{frame}
