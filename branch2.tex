% This is part of the ``Git First Step''.
% Copyright 2014, 2020 Yoshida Shin.
% See the file slide.tex for copying conditions.

\begin{frame}<1-3>[t]{branch (削除)}{branch を使用する}

  既存の branch を削除
  \vspace{4ex}

  \onslide*<1>{
    現在の branch 一覧
    \vspace{2ex}

    \$ git branch

    \code{
      \onslide<99>{* }develop

      * master
    }
  }

  \onslide*<2>{
    branch 削除
    \vspace{2ex}

    \$ git branch -d develop
  }

  \onslide*<3>{
    現在の branch 一覧
    \vspace{2ex}

    \$ git branch

    \code{
      * master
    }
  }

\end{frame}

\begin{frame}<1-5>[t]{中で起きている事}{branch を使用する}

  \begin{columns}

    \begin{narrowcolumn}

      \begin{block}{workspace}
        c d e
      \end{block}

      \begin{block}{cached (index)}
        c d e
      \end{block}

    \end{narrowcolumn}

    \begin{widecolumn}

      \begin{leftrepository}{repository}
        \node (c5) [commit, rectangle split, rectangle split parts=2] at (0em, 14ex){
          \commitmessage{Merge branch 'develop'}
          \nodepart{second}{master{\HEAD}}
        };

        \onslide*<3->{
          \node (c4) [commit, rectangle split, rectangle split parts=1] at (5em, 6.5ex){
            \commitmessage{Add e.}
          };
        }

        \onslide*<-3>{
          \node (c4) [commit, rectangle split, rectangle split parts=2] at (5em, 6ex){
            \commitmessage{Add e.}
            \nodepart{second}{develop}
          };
        }

        \node (c3) [commit, rectangle split, rectangle split parts=1] at (0em, 6.5ex){
          \commitmessage{Add c.}
        };

        \node (c2) [commit, rectangle split, rectangle split parts=1] at (0em, 0ex){
          \commitmessage{Rename a to d and delete b.}
        };

        \draw (c4) -- (c5);
        \draw (c3) -- (c5);
        \draw (c2) -- (c4);
        \draw (c2) -- (c3);
      \end{leftrepository}

    \end{widecolumn}

  \end{columns}
  \vspace{2ex}

  \onslide*<2>{
    \$ git branch
    \vspace{2ex}

    \code{
      \onslide<99>{* }develop

      * master
    }
  }

  \onslide*<3-4>{
    \$ git branch -d develop
    \vspace{2ex}

    develop branch を削除

    (commit は消えない)
  }

  \onslide*<5>{
    \$ git branch
    \vspace{2ex}

    \code{
      * master
    }
  }

\end{frame}


\begin{frame}[t]{branch (削除)}{branch を使用する}

  \begin{itemize}
  \item git branch -d \textit{branch1} [\textit{branch2} [\textit{branch3}] ...]

    で複数 branch の削除が可能
    \vspace{2ex}

  \item branch を削除しても、commit は消えない
     \vspace{2ex}

  \item branch を削除すると、その commit が

    見つけにくくなる場合がある\footnote{後述}

    そのような場合、誤操作防止の為に

    git branch -d はエラーとなる
    \vspace{2ex}

  \item git branch -D を使用すると、上記の場合でも

    branch の強制削除可能
  \end{itemize}

\end{frame}
