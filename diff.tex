% This is part of the ``Git First Step''.
% Copyright 2014, 2020 Yoshida Shin.
% See the file slide.tex for copying conditions.

\begin{frame}[t]{diff}{レポジトリの状態を確認する}

  workspace, cached, commit 間の差分をパッチレベルで表示
  \vspace{4ex}

  \onslide*<1>{
    \$ git diff
    \vspace{2ex}

    workspace と cached の差分を表示

    (差分は無いので何も表示されない)
  }

  \onslide*<2>{
    \$ git diff {\dhyphen}cached
    \vspace{2ex}

    cached と HEAD の差分を表示
    \vspace{2ex}

    \code{
      diff {\dhyphen}git a/f b/f

      new file mode 100644

      index 0000000..9daeafb

      {\hyphen}{\hyphen}{\hyphen} /dev/null

      +++ b/f

      @@ -0,0 +1 @@

      +test
    }
  }

  \onslide*<3>{
    \$ git diff 3658
    \vspace{2ex}

    workspace と commit の差分を表示
    \vspace{2ex}

    \code{
      diff {\dhyphen}git a/c b/c

      new file mode 100644

      index 0000000..e69de29

      diff {\dhyphen}git a/f b/f

      new file mode 100644

      index 0000000..9daeafb

      {\hyphen}{\hyphen}{\hyphen} /dev/null

      +++ b/f

      @@ -0,0 +1 @@

      +test
    }
  }

  \onslide*<4>{
    \$ git diff 3658 HEAD
    \vspace{2ex}

    3658 と HEAD の差分を表示
    \vspace{2ex}

    \code{
      diff {\dhyphen}git a/c b/c

      new file mode 100644

      index 0000000..e69de29
    }
  }

\end{frame}


\begin{frame}[t]{git diff コマンド補足}{レポジトリの状態を確認する}

  \begin{itemize}
  \item git diff [{\dhyphen}]\footnote{曖昧さ回避の為のオプション、通常は必要ない} [\textit{path1} [\textit{path2} [...]]]\footnote{path が省略されると、git 管理下の全ファイル}

    で path の workspace と cached の差分を表示
    \vspace{2ex}

  \item git diff \textit{commit} [{\dhyphen}] [\textit{path1} [\textit{path2} [...]]]

    で path の workspace と commit の差分を表示

    commit は HEAD や branch, ハッシュ値等で指定
    \vspace{2ex}

  \item git diff {\dhyphen}cached \textit{commit} [{\dhyphen}] [\textit{path1} [\textit{path2} [...]]]

    で path の commit と cached の差分を表示
    \vspace{2ex}

  \item git diff \textit{commit1} \textit{commit2} [{\dhyphen}] [\textit{path1} [\textit{path2} [...]]]

    で path の commit 間の差分を表示
  \end{itemize}

\end{frame}
