% This is part of the ``Git First Step''.
% Copyright 2014, 2020 Yoshida Shin.
% See the file slide.tex for copying conditions.

\begin{frame}[t]{diff}{レポジトリの状態を確認する}
  git diff は workspace, cached, commit 間の差分をパッチレベルで表示
  \vspace{4ex}

  「何の差分を見るか」を指定するオプションが少し複雑
  \vspace{4ex}

  どの状態とどの状態の差分か?
  \begin{itemize}
  \item workspace
  \item cached
  \item commit
  \end{itemize}
  (デフォルトでは workspace と cached の差分)

  \vspace{2ex}

  どのファイルの差分か?

  (デフォルトでは git が管理下の全ファイルの差分)
\end{frame}


\begin{frame}[t]{diff}{レポジトリの状態を確認する}

  どの状態とどの状態の差分か?
  \vspace{4ex}

  \begin{itemize}
  \item 指定が無いと、workspace と cached の差分
  \item 1 個だけ指定すると、workspace と指定した状態の差分
  \item 2 個指定すると、指定した状態間の差分
  \end{itemize}
\end{frame}


\begin{frame}[t]{中で起きている事}{レポジトリの状態を確認する}

  \begin{columns}

    \begin{narrowcolumn}

      \begin{block}{workspace}
        c d e f g
      \end{block}

      \begin{block}{cached (index)}
        c d e f
      \end{block}

    \end{narrowcolumn}

    \begin{widecolumn}

      \begin{repository}{repository}
        \node (c1) [commit, rectangle split, rectangle split parts=3] at (7em, 11ex) {
          \commitmessage{Merge branch 'develop'}
          \nodepart{second}{c d e}
          \nodepart{third}{master{\HEAD}}
        };

        \node (c2) [commit, rectangle split, rectangle split parts=3] at (9em, -1ex){
          \commitmessage{Add e.}
          \nodepart{second}{d e}
          \nodepart{third}{3658...}
        };

        \node (c3) [commit, rectangle split, rectangle split parts=1] at (0em, -2ex){
          \commitmessage{Add c.}
        };

        \draw (c1) -- (c2);
        \draw (c1) -- (c3);

      \end{repository}

    \end{widecolumn}

  \end{columns}
  \vspace{2ex}

  HEAD と cached の差分: f の有無

  workspace と cached の差分: g の有無
\end{frame}


\begin{frame}[t]{diff}{レポジトリの状態を確認する}

  \onslide*<1>{
    \$ git diff
    \vspace{2ex}

    workspace と cached の差分を表示
    \vspace{2ex}

    今回は差分は無いので何も表示されない

    (g は git 管理下に無いので無視される)
  }

  \onslide*<2>{
    \$ git diff {\dhyphen}cached
    \vspace{2ex}

    cached と HEAD の差分を表示
    \vspace{2ex}

    \code{
      diff {\dhyphen}git a/f b/f

      new file mode 100644

      index 0000000..9daeafb

      {\hyphen}{\hyphen}{\hyphen} /dev/null

      +++ b/f

      @@ -0,0 +1 @@

      +test
    }
  }

  \onslide*<3>{
    \$ git diff 3658
    \vspace{2ex}

    workspace と commit の差分を表示
    \vspace{2ex}

    \code{
      diff {\dhyphen}git a/c b/c

      new file mode 100644

      index 0000000..e69de29

      diff {\dhyphen}git a/f b/f

      new file mode 100644

      index 0000000..9daeafb

      {\hyphen}{\hyphen}{\hyphen} /dev/null

      +++ b/f

      @@ -0,0 +1 @@

      +test
    }
  }

  \onslide*<4>{
    \$ git diff 3658 HEAD
    \vspace{2ex}

    3658 と HEAD の差分を表示
    \vspace{2ex}

    \code{
      diff {\dhyphen}git a/c b/c

      new file mode 100644

      index 0000000..e69de29
    }
  }

\end{frame}


\begin{frame}[t]{git diff コマンド補足}{レポジトリの状態を確認する}

  \begin{itemize}
  \item git diff [{\dhyphen}]\footnote{曖昧さ回避の為のオプション、通常は必要ない} [\textit{path1} [\textit{path2} [...]]]\footnote{path が省略されると、git 管理下の全ファイル}

    で path の workspace と cached の差分を表示
    \vspace{2ex}

  \item git diff \textit{commit} [{\dhyphen}] [\textit{path1} [\textit{path2} [...]]]

    で path の workspace と commit の差分を表示

    commit は HEAD や branch, ハッシュ値等で指定
    \vspace{2ex}

  \item git diff {\dhyphen}cached \textit{commit} [{\dhyphen}] [\textit{path1} [\textit{path2} [...]]]

    で path の commit と cached の差分を表示
    \vspace{2ex}

  \item git diff \textit{commit1} \textit{commit2} [{\dhyphen}] [\textit{path1} [\textit{path2} [...]]]

    で path の commit 間の差分を表示
  \end{itemize}

\end{frame}
