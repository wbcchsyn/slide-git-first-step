% This is part of the ``Git First Step''.
% Copyright 2014, 2020 Yoshida Shin.
% See the file slide.tex for copying conditions.

\begin{frame}[t]{init (ローカル)}{使ってみる}

  git を使い始める準備をする
  \vspace{4ex}

  \$ mkdir \~{}/local\_rep

  \$ cd \~{}/local\_rep\footnote{この後、しばらくは \~{}/local\_rep で作業します}

  \$ git init

  \$ git config user.name 'Your Name'

  \$ git config user.email 'your@address'

\end{frame}

\begin{frame}[t]{中で起きている事}{使ってみる}

  \begin{columns}

    \begin{narrowcolumn}

      \onslide<2->{
        \begin{block}{workspace}
        \end{block}
      }

      \onslide<5->{
        \begin{block}{cached (index)}
        \end{block}
      }

    \end{narrowcolumn}

    \begin{halfcolumn}

      \onslide<5->{
        \begin{block}{repository}
        \end{block}
      }

    \end{halfcolumn}

  \end{columns}
  \vspace{2ex}

  \onslide*<1-2>{
    \$ mkdir \~{}/local\_rep

    \$ cd \~{}/local\_rep
    \vspace{2ex}

    ここで作成するディレクトリが workspace
  }

  \onslide*<3-5>{
    \$ git init
    \vspace{2ex}

    \onslide*<4-5>{ワークスペース直下に .git ディレクトリを作成し、その中に cached や repository を初め git に必要な物を作成する}
    \onslide*<6>{今後、.git ディレクトリを直接触るか、git 以外のコマンドは、基本的に workspace の作業と思ってよい。}
  }

  \onslide*<6>{
    \$ git config user.name 'Your Name'

    \$ git config user.email 'your@address'
    \vspace{2ex}

    この git の設定ファイルを編集

    workspace, cached, repository には影響無し
  }

\end{frame}
