% This is part of the ``Git First Step''.
% Copyright 2014, 2020 Yoshida Shin.
% See the file slide.tex for copying conditions.

\begin{frame}[t]{checkout}{過去の履歴を活用する}

  ``git reset'' では cached を変更せずに workspace を cached に揃える事ができない

  \vspace{2ex}

  \begin{itemize}
  \item git reset --mixed

    cached を repository に揃える
  \item git reset --hard

    workspace と cached を repository に揃える
  \end{itemize}

  \vspace{2ex}

  そのような事をしたい場合も git checkout で行う

  (git checkout の 2 個目の機能)

\end{frame}


\begin{frame}[t]{checkout}{過去の履歴を活用する}

  workspace と cached の状態を変化させ、最終的に元に戻す

  \vspace{4ex}

  \$ echo 'c' \textgreater \space c

  \$ git add c

  \vspace{2ex}

  \$ echo '\_c\_' \textgreater \space c

  \$ git checkout c

  \vspace{2ex}

  \$ git reset {\dhyphen}soft

  \$ git checkout c

\end{frame}


\begin{frame}[t]{中で起きている事}{過去の履歴を活用する}

  \begin{columns}

    \begin{narrowcolumn}

      \begin{block}{workspace}
        \onslide*<1,8>{\color{black}} \onslide*<2,5-7>{\color{red}} \onslide*<3-4>{\color{blue}} c
        \color{black} d e
      \end{block}

      \begin{block}{cached (index)}
        \onslide*<1,6->{\color{black}} \onslide*<2-5>{\color{red}} c
        \color{black} d e
      \end{block}

    \end{narrowcolumn}

    \begin{widecolumn}

      \begin{repository}{repository}
        \node (c1) [commit, rectangle split, rectangle split parts=3] at (0em, 0ex) {
          \commitmessage{Merge branch 'develop'}
          \nodepart{second}{c d e}
          \nodepart{third}{master{\HEAD}}
        };
      \end{repository}

    \end{widecolumn}

  \end{columns}
  \vspace{2ex}

  \onslide*<2>{
    \$ echo 'c' \textgreater \space c

    \$ git add c

    \vspace{2ex}

    workspace と cached が変更される
  }

  \onslide*<3> {
    \$ echo '\_c\_' \textgreater \space c

    \vspace{2ex}

    workspace のみ変更される
  }

  \onslide*<4-5>{
    \$ git checkout c

    \onslide<5>{
      \vspace{2ex}

      workspace が cached の状態に戻る
    }
  }

  \onslide*<6>{
    \$ git reset {\dhyphen}soft

    \vspace{2ex}

    cached が repository の状態に戻る
  }

  \onslide*<7-8>{
    \$ git checkout c

    \onslide<8>{
      \vspace{2ex}

      workspace が cached の状態に戻る
    }
  }

\end{frame}


\begin{frame}[t]{git checkout コマンド補足}{過去の履歴を活用する}

  \begin{itemize}
  \item git checkout \textit{commit}

    HEAD の commit へ切り替え

    \vspace{2ex}

  \item git checkout [{\dhyphen}]  \textit{path1} [\textit{path2} [\textit{path3} [...]]]

    workspace の複数の path を cached の状態に戻す

    \vspace{2ex}

    path がディレクトリの場合、そのディレクトリ以下のファイルを再帰的に git checkout する

    \vspace{2ex}

    {\dhyphen} は path と同名の branch があった場合に、1 個目の使い方と区別するために使用する

    (滅多に使用しない)
  \end{itemize}

\end{frame}
