% This is part of the ``Git First Step''.
% Copyright 2014, 2020 Yoshida Shin.
% See the file slide.tex for copying conditions.

\begin{frame}[t]{push}{使ってみる}
  local branch のコピーを git server に作成
  \vspace{4 ex}

  \onslide*<1>{
    master branch を転送
    \vspace{2ex}

    \$ git push origin master
  }

  \onslide*<2>{
    develop branch 作成
    \vspace{2ex}

    \$ git branch develop
  }

  \onslide*<3>{
    master branch の更新

    (git add f は git status の説明時に実施済み)
    \vspace{2ex}

    \$ git commit -m 'Add f.'
  }

  \onslide*<4>{
    master branch の再転送
    \vspace{2ex}

    \$ git push origin master
  }

  \onslide*<5>{
    develop branch の転送
    \vspace{2ex}

    \$ git push origin develop
  }
\end{frame}


\begin{frame}[t]{中で起きている事}{使ってみる}

  \begin{columns}

    \begin{halfcolumn}

      \begin{repository}{local}
        \onslide<8->{
          \node (c6) [commit, rectangle split, rectangle split parts=2] at (0em, 15ex){
            \commitmessage{Add f.}
            \nodepart{second}{master{\HEAD}}
          };
        }

        \node (c5) [commit, rectangle split, rectangle split parts=2] at (0em, 6ex){
          \commitmessage{Merge branch 'develop'}
          \nodepart{second}{\onslide*<-7>{master{\HEAD}} \onslide*<6->{develop}}
        };

        \node (c4) [commit, rectangle split, rectangle split parts=1] at (2em, 0ex){
          \commitmessage{Add e.}
        };

        \node (c3) [commit, rectangle split, rectangle split parts=1] at (-2em, 0ex){
          \commitmessage{Add c.}
        };

        \onslide*<8->{\draw (c5) -- (c6);}
        \draw (c4) -- (c5);
        \draw (c3) -- (c5);
      \end{repository}

    \end{halfcolumn}

    \begin{halfcolumn}

      \begin{repository}{remote}
        \onslide<12->{
          \node (c6) [commit, rectangle split, rectangle split parts=2] at (0em, 15ex){
            \commitmessage{Add f.}
            \nodepart{second}{master}
          };
        }

        \onslide<4->{
          \node (c5) [commit, rectangle split, rectangle split parts=2] at (0em, 6ex){
            \commitmessage{Merge branch 'develop'}
            \nodepart{second}{\onslide*<-11>{master}\onslide*<15->{develop}}
          };

          \node (c4) [commit, rectangle split, rectangle split parts=1] at (2em, 0ex){
            \commitmessage{Add e.}
          };

          \node (c3) [commit, rectangle split, rectangle split parts=1] at (-2em, 0ex){
            \commitmessage{Add c.}
          };

          \draw (c4) -- (c5);
          \draw (c3) -- (c5);
        }

        \onslide<12->{\draw (c5) -- (c6);}
      \end{repository}

    \end{halfcolumn}

  \end{columns}
  \vspace{2ex}

  \onslide*<2-4>{
    \$ git push origin master
    \vspace{2ex}

    \onslide*<2>{origin に master branch が有るか確認 (無い)}

    \onslide*<3-4>{origin に master branch の指す commit をコピー}
  }

  \onslide*<5-6>{\$ git branch develop}

  \onslide*<7-8>{\$ git commit -m 'Add f.'}

  \onslide*<9-12>{
    \$ git push origin master
    \vspace{2ex}

    \onslide*<9>{origin に master branch が有るか確認 (有る)}

    \onslide*<10>{
      origin の master branch と同じ commit が

      local の master branch の 先祖か確認 (先祖である)
    }

    \onslide*<11-12>{

      origin に master branch をコピー

      (新しい commit, ファイルの転送と master branch の移動)
    }
  }

  \onslide*<13-15>{
    \$ git push origin develop
    \vspace{2ex}

    \onslide*<13>{origin に develop branch が有るか確認 (無い)}

    \onslide*<14-15>{

      origin に develop branch をコピー

      (branch の作成のみ)
    }
  }

\end{frame}


\begin{frame}[t]{git push コマンド補足}{使ってみる}

  \begin{itemize}
  \item git push \textit{URI} \textit{source-branch}[:\textit{destination-branch}]

    で source-branch を URI の destination-branch にコピー

    destination-branch のデフォルトは source-branch
    \vspace{2ex}

  \item git push \textit{URI} :\textit{destination-branch}

    で URI の destination-branch を削除

    (空の branch をコピーすると思えば良い)

    remote repository の branch を削除しても local repository の branch は残る
    \vspace{2ex}

  \item commit のコピーとは、commit 自身, その先祖の全 commit, 各 commit で必要な全ファイル、ディレクトリのコピー
  \end{itemize}

\end{frame}
